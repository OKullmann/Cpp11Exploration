\documentclass[11pt,]{report}

\usepackage{a4}
\usepackage{graphicx}
\usepackage{hyperref}

% For program-code:
\usepackage{listings}
\lstloadlanguages{C++,make}
\newcommand{\Cpp}{\lstset{language=C++,keywordstyle=\bfseries,breaklines,breakindent=30pt}}
\newcommand{\Make}{\lstset{language=make}}
\newcommand{\inl}[1]{\lstinline$#1$}

\Cpp


\begin{document}

\author{Dilshad Sallo\\[1ex]
  Computer Science Department\\
  College of Science, Swansea University\\
  Swansea, SA2 8PP, UK\\[1ex]
  Student Number: 59950\\
  Under Supervision of Dr. Oliver Kullmann
}

\title{C++11 by Examples}

\maketitle



\begin{abstract}

C++11 is new standard that released by C++ committee standard \linebreak representing the effort of most expert in the world. This version includes new features that addresses some limitations exist in traditional C++ as well as \linebreak providing new features that never existed in any previous versions. These \linebreak features support a core language and some libraries and  provide new facilities  to make programming easier than before.

The aim of this project is to investigate some of these features (core \linebreak language features) and providing examples to get most out of them and show them as scientific catalogue to support  ordinary programmers.

This paper focuses on investigating core language features which classify to categories depending in which they belong to. Then, it clearly explains the structure  and  approach of achievement the project which should be mentioned in order to explain how the project will be implemented in the  summer. Finally, it should not be forgotten the important part of this report is the time management which  I use it to arrange the time in the summer to reach the purpose of project.
\end{abstract}

\tableofcontents


\chapter{Introduction}
\label{cha:Introduction}

\section{Introduction}  
\label{section: intorduction}

C++ is a general-purpose language that is in widespread use for systems and embedded programming. It is a high-level programming language that allows a programmer to efficiently communicate with a computer. C++ fully supports object-oriented programming, including the four pillars of object-oriented development, namely Encapsulation, Data hiding, Inheritance and Polymorphism.

Standard C++ consists of three important parts; the core language giving all the building blocks including variables, data types and literals, the C++ Standard Library giving a rich set of functions manipulating files, strings, and the Standard Template Library (STL) giving a rich set of methods manipulating data structures.

Due to complexity applications  that required in various fields and evolution of programming that  accompanied by modern technology, some limitations and bugs have discovered in C++ language that have significant impact on \linebreak programmers of all backgrounds.

As responsibility of C++ Standards Committee to address this issue, new standard is released dubbed C++11. It is represented by new features that added to core language, Standard  Library, and Standard Template Library. The purposes of these features are make programming easier by providing simple syntax, providing new style of programming and adding new facilities that never existed in any previous versions. These lead to support ordinary programmers to write robust codes using efficient style.

The goal of this report is showing how core language features will be \linebreak investigated to support ordinary  programmers. They represent by clear \linebreak structure of achievement that written in formal methods, which represent some categories that  classify depending on features belong to. The report also show how does the project achieve  which depend on the nature of features. Finally, the report presents the plan which will be implemented in the summer, and this plan has divided to days  which implement the features step by step to get in the final form.


\section{The Purpose of Project}
\label{section:The purpose of project}
The aim of this project is to investigate core language  features of new \linebreak standard (dubbed C++11) carefully, and presenting them as examples for \linebreak ordinary programmers who have experience in C++, to understanding this new style of programming languages by using them as scientific  catalogue to develop their skills, and shift toward C++11 language.

On the completion of this project, programmers who use this project, will get confidence to understand core language of  C++11 and use it with \linebreak elementary applications.

\section{Objectives of Project}
\label{section:objectives of project}
In order to achieve the aim of project, multiple objectives should be followed and done correctly,they are:-
 \begin{enumerate}
   \item Complete a search about core language features and their definitions.
   \item Identify cases of usage for each feature  to know its functionality.
   \item Presenting each case by simple example to make understanding of \linebreak programmers easier.
   \item Show all examples as scientific catalogue to support ordinary  programmers.
   \item Complete final report.
 \end{enumerate}       
 
  
\section{History}
\label{section: History}
The history of C++ programming language has begun in 1979, when Bjarne Stroustrup was working on his Ph.D thesis in the Computing Laboratory of Cambridge, University of England. At that time, he had opportunity to work with a language called Simula, which is considered as the first language to support the object-oriented language paradigm. He observed that paradigm of Simula was very useful for software development\cite{StroustrupHistory}.

Shortly after that, he started work on "C with Classes” to accomplish his goal, which represent adding object-oriented programming into the C language with respected all its functionality. His language included basic inheritance, inlining functions, classes, default function arguments, and strong type checking as well as including all the features of the C language\cite{StroustrupHistory}.

In 1983, the name of C with Classes was change to C++.  The ++ operator in the C language refer to incrementing a variable, which gives some understanding into how Stroustrup considered the language. During that time, many important new features added to the language such as  virtual functions, such as function overloading, references with the \& symbol, the const keyword, and single-line comments using two forward slashes\cite{StroustrupHistory}.

In 1985, Stroustrup's reference to the language titled The C++  Programming Language was published. C++ was carried out as a  commercial product in the same year though it was not officially standardized. The language was updated again in 1989 to include protected and static members, as well as inheritance from several classes\cite{StroustrupHistory}.

In 1990, the Annotated C++ Reference Manual was released, at that time, Borland's Turbo C++ compiler was also released as a commercial product. Turbo C++ added some additional libraries, which could have significant \linebreak impact on C++'s development\cite{StroustrupHistory}.

In 1998, the C++ standards committee issued the first international \linebreak standard for C++ ISO/IEC 14882:1998, which would be informally known as C++98. The Annotated C++ Reference Manual has an important influence on the development of the standard.  In addition it was included the Standard Template Library which started its conceptual development in 1979. Due to multiple problems that discovered in 1998 standard, the committee revised it and changed language was dubbed C++0x in 2003\cite{CplusplusHistoryofCpp}.

In 2005, the C++ standards committee published a technical report (called TR1) detailing several features which expected to add to the latest C++ standard by committee. Later on, the new standard was informally called C++0x.  The first draft for formal comments was produced in September 2008 and then the Final International Draft standard unanimously approved by the ISO C++ committee in 2011. It was formally approved (dubbed C++11)\cite{CplusplusHistoryofCpp}.

The new C++ standard (C++11) was published in 2011. The Boost library project made a significant effect on the new standard and it also derived some of new modules from other libraries. As the result, some new features included in the new standard, such as regular expression support, atomics support, a standard threading library which never existed in both C and previous C++, a new for loop syntax with similar functionality for foreach loops in other languages, the auto keyword, new container classes and variadic templates\cite{CplusplusHistoryofCpp}. 


\chapter{Literature Review}

\section{C++}
\label{section: C++}
C++ is most popular language which has important role in the industry, since it use for just about everything such as building compilers and run-times for competing languages, and operating systems like Windows. In addition, it uses in many social web applications such as browse Google Chrome and face book \cite{ISO:2011:Cpplanguage}.

C++ bring together three separate programming categories: the procedural language, represented by C; the object-oriented language, represented by the class enhancements C++ adds to C; and generic programming, supported by C++ templates\cite{Prata:2012:Cpp}.

One reason that made C++ most popular, is that, object-oriented programming which emphasizes the data rather than procedural programming, which emphasizes algorithms, but with modern applications which use concurrent, distributed programming and complicated codes, programmers need to think differently about system design and implementation as well as using easy syntax  to create robust code.This can be achieved with new standard language C++11\cite{Stroustrup:2012:Cpp11}.

\section{C++11}
\label{section: C++11}
C++11 is the new standard of C++ language. It was known as C++0x until mid- 2011, and then formally published as ISO/IEC 14882:2011. C++11 feels like a new language and a higher-level style of programming more efficient than before\cite{ISO:2011:Cpplanguage}. 

C++11 provide new facilities(features) which make programming more \linebreak natural and efficient than before as well as helping programmers to think \linebreak differently about system design and implementation \cite{Stroustrup:2012:Cpp11}. This is not just for C++ programmers but also for programmers who used to programming with modern languages in general areas. Moreover, it improves abstraction \linebreak mechanisms and makes it more flexible and safely by using some features \linebreak together. These features extend the powerful of traditional C++ in terms of flexibility and efficiency as well as enabling programmers to get most benefit of them to write robust codes \cite{ISO:2011:Cpplanguage}.

C++11 adds many new language features to C++ that should cover some limitation and reduce the overall verbosity of C++ as well as provide new concept, such as lambda expressions, that increase its overall expressiveness and clarity.

C++11 provides features such as auto keywords to make C++ more usable language as well as features like move semantics to improve the basic efficiency of the language, allowing programmer to write faster code, and the improvement to the template system make it much easier to write generic code\cite{Stroustrup:2005:Cpp}.

C++11 also adds new features to the standard library which never existed in any previous versions, including adding multi-threading support directly into C++ and improved smart pointers that will simplify memory management. 

In general, these features have different behaviour, since some features may make programming a little bit difficult because they replace some rules with one more general rule such as uniform initialization,  and inheriting constructors, others easier by providing more flexible features than older version such as auto, array and range-for-statement \cite{Stroustrup:2012:Cpp11}.

	

\section{Boost Library}
\label{section: Boost Library}
The Boost C++ Libraries are free, open source libraries created by members of the C++ community. Boost provides useful, well-designed libraries that work well with the existing C++ Standard Library. Boost can be used on many platforms with many different compilers. Ten Boost libraries are included in the C++ Standards Committee's Library Technical Report (TR1). These \linebreak libraries add useful functionality to C++.  C++11 also includes several more Boost libraries in addition to those from TR1, namely, boost.array, boost.bind, boost.function, boost.random, boost.regex, boost.smart\_ptr, boost.tuple and boost.type\_traits. These libraries consider a part of C++11 and they provide many facilities that support programmers\cite{Deitel:2012:CPP}.


\section{Why C++11?}
\label{secton: Why C++11}
The nature of C++ focus on general features (notably classes) to demonstrate its main strength, while it has main weakness that represent by specialized features (such as threads and properties) are neglected.

The main goals for C++11 are to make C++ easier to learn, teach, and understand, improve library building capabilities, and increase compatibility with the existing standard C++ and C programming language \cite{Deitel:2012:CPP}.

In addition, while traditional C++ is not only designed for Windows \linebreak language or Web language but also for general purpose, C++11 will extend that and support generic programming, concurrent, parallel and distributed programming. 
The new features can save programmers from fading into  dark corners  and make through less error-prone language features and through more supportive libraries.


\section{Why Small and Many Programs?}
\label{section: why Small and Many programs}
knowing the syntax of C++11 language is not enough to learn and understand the correct way to use the language, unless providing  simple  programs that give idea to programmers from first glance. This can be achieved by small and many programs that represent features and their usages.


\section{Core Language}
\label{section: core language}

\subsection{Core Language Runtime Performance Enhancements}
\begin{enumerate}
\item \textbf{Rvalue references:} 
\item \textbf{Move semantic:}
\item \textbf{Generalized constant expressions- constexpr keyword:} 
\item \textbf{Modification of the definition of plain old data}
\end{enumerate}


\subsection{Core Language Usability Enhancements}
\begin{enumerate}
\item \textbf{Uniform initialization:}
\item \textbf{Initializer lists:}
\item \textbf{Type inference - Auto keyword:} 
\item \textbf{Type inference – Decltype keyword:} 
\item \textbf{Range-based for statement:} 
\item \textbf{Alternative function syntax:}
\item \textbf{Lambda functions:} 
\item \textbf{Alternative function syntax:}
\item \textbf{Delegating constructors: }
\item \textbf{Override keyword:}
\item \textbf{Final keyword:}
\item \textbf{Null pointer constant:}
\item \textbf{Strongly typed enumerations:}
\item \textbf{Alias templates:}
\end{enumerate}


\subsection{Core Language Functionality Improvements}
\begin{enumerate}
\item \textbf{New character types : char16\_t and char32\_t}
\item \textbf{Raw string literals:}
\item \textbf{User-defined literals:}
\item \textbf{Explicitly Defaulted special member functions:}
\item \textbf{Explicitly Deleted special member functions:}
\item \textbf{Unsigned long long (int) and long long (int):} 
\item \textbf{Static assertion:}
\item \textbf{Variadic template:}
\end{enumerate}



\section{Summary}
\label{section: summary}

To sum up, C++11 supports ordinary programmers by offers much more than (old) C++  and  providing new features which have significant role to enhance memory performance, provide flexibility to solve important problems and make C++ language easier to understand. It also not introduce new holes and  ensure to avoiding unsafe facilities that existed in previous version.

\addcontentsline{toc}{section}{Bibliography}
\bibliographystyle{alpha}
\bibliography{Bibliography}	


\begin{appendix}

\chapter{Program Code}
\label{chapter:Programcode}


\section{MakeFile}
\label{Makefile}

\Make

\lstinputlisting{../Makefile.}
\newpage

% Core language Runtime Performance Enhancements
\section{Core Language Runtime Performance Enhancements}
\label{Appendix: corelanguage runtime performance}

\Cpp

\subsection{RvalueReference\_Basic.cpp}
\label{sub:RvalueReference_Basic}
\lstinputlisting{../CoreLanguage/RunTime/RvalueReference_Basic.cpp}


% Core Language Usability Enhancements
\section{Core Language Usability Enhancements}
\label{Appendix: corelanguage usabiliy enhancements}

\Cpp

\subsection{}
\label{}
%\lstinputlisting{../CoreLanguage/UsabilityEnhancements/}


% Core Language Functionality Improvements
\section{Core Language Functionality Improvements}
\label{Appendix: corelanguage functionality improvements}

\Cpp


\subsection{}
\label{}
%\lstinputlisting{../CoreLanguage/FunctionalityImprovements/


\end{appendix}

\end{document}
